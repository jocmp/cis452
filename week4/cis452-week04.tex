% !TEX program = pdflatex
\documentclass[11pt]{article}
\usepackage{palatino}
\usepackage{fullpage}
\usepackage{listings}
\usepackage{url}
%
% Shell command markup
\newcommand{\shellcmd}[1]{\texttt{\footnotesize #1}}

% Rename this file to cis452-weekNN.tex
% To produce a PDF output, type the following on EOS
%
%  pdflatex cis452-weekNN.tex
%
\title{CIS452 Week-03 Notes}
\author{Winter 2016}
\date{February 1-5, 2016}
\begin{document}
\maketitle

\noindent
\textbf{Scribes:} \textit{Josiah Campbell} and \textit{Kevin Tarquinio}

\section*{Announcements}
\begin{itemize}
  \item Homework Chapter 4, due February 5th
\end{itemize}

% Monday 2.1.2016
\section*{Design Issues}
\begin{itemize}
\item fork()
  \\Two options:
  \begin{enumerate}
  \item Fork will always run in a single thread
  \item Clone (child) will always run multithreaded
  \end{enumerate}
\item exec() - load a new binary executable then... do something\\
  It is uncommon to call exec \textbf{during} your own program. You call exec
  after fork
  \begin{enumerate}
  \item exec will create a new thread
  \item exec will act in only one thread even if the thread library runs
  \end{enumerate}
\item signal() - the solution is dependent
  \begin{itemize}
      \item Delivered asynchronously (mostly)
      \begin{itemize}
          \item Example: SIGINT using Ctrl-C is async $\rightarrow$ you never
          know when the system will deliver the call
      \end{itemize}
      \item Some signals are delivered synchronously
      \begin{itemize}
          \item Example: SIGFPE - floating point errors
          \item Other arithmetic errors
          \item Call comes \textit{right after} you do the incorrect thing
      \end{itemize}
  \end{itemize}
\end{itemize}

% Wednesday notes
\section*{Threading in different languages}
Happy examples: \url{https://gist.github.com/dulimarta/a7e25cb3984e5d2b8e16}


%Friday notes
\section*{Threads Task parallelism, Data Parallelism, Threading Implementations}

\subsection*{Design for parallelism}
\begin{enumerate}
  \item Task Parallelism
  \begin{itemize}
    \item The program is modularized so that unique modules run on different CPU's
	\item Modules consist of different instructions
	\item Example: A video player with a different modules for audio, UI,
	        rendering, and buffering is an example.
  \end{itemize}
  \item Data Parallelism
	 \begin{itemize}
	   \item Deploy smaller subsets of data on multiple CPUs in parallel to
	        partition the processing workload
	   \item The same instructions/algorithm is used on each processor but
	        with smaller data sets (SPMD = Single Program Multiple Data)
	  \end{itemize}
\end{enumerate}


\subsection*{ Thread State Diagram}
	  \begin{itemize}
	  \item The thread state diagram is the same as the process state diagram except that a process with multiple threads has a process state diagram with a containing a thread state diagram for every thread.
	    \begin{itemize}
	        \item aaa
	    \end{itemize}
	  \end{itemize}
\end{document}
