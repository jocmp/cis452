% !TEX program = pdflatex
\documentclass[11pt]{article}
\usepackage{palatino}
\usepackage{fullpage}
\usepackage{listings}
\usepackage{color}
\usepackage{url}
%
% Shell command markup
\newcommand{\shellcmd}[1]{\texttt{\footnotesize #1}}

\definecolor{mygreen}{rgb}{0,0.6,0}
\definecolor{mygray}{rgb}{0.5,0.5,0.5}
\definecolor{mymauve}{rgb}{0.58,0,0.82}
% Source: Overleaf - https://goo.gl/uL1le2
\lstset{ %
  backgroundcolor=\color{white},   % choose the background color
  basicstyle=\footnotesize,        % size of fonts used for the code
  breaklines=true,                 % automatic line breaking only at whitespace
  captionpos=b,                    % sets the caption-position to bottom
  commentstyle=\color{mygreen},    % comment style
  escapeinside={\%*}{*)},          % if you want to add LaTeX within your code
  keywordstyle=\color{blue},       % keyword style
  stringstyle=\color{mymauve},     % string literal style
}


% Rename this file to cis452-weekNN.tex
% To produce a PDF output, type the following on EOS
%
%  pdflatex cis452-week04.tex
%
\title{CIS452 Week-04 Notes}
\author{Winter 2016}
\date{February 1-5, 2016}
\begin{document}
\maketitle

\noindent
\textbf{Scribes:} \textit{Josiah Campbell} and \textit{Kevin Tarquinio}

\section*{Announcements}
\begin{itemize}
  \item Homework Chapter 4, due February 5th
\end{itemize}

% Monday 2016-02-01
\section*{Design Issues}
\begin{itemize}
\item fork()
  \\Two options:
  \begin{enumerate}
  \item Fork will always run in a single thread
  \item Clone (child) will always run multithreaded
  \end{enumerate}
\item exec() - load a new binary executable then... do something\\
  It is uncommon to call exec \textbf{during} your own program. You call exec
  after fork
  \begin{enumerate}
  \item exec will create a new thread
  \item exec will act in only one thread even if the thread library runs
  \end{enumerate}
\item signal() - the solution is dependent
  \begin{itemize}
      \item Delivered asynchronously (mostly)
      \begin{itemize}
          \item Example: SIGINT using Ctrl-C is async $\rightarrow$ you never
          know when the system will deliver the call
      \end{itemize}
      \item Some signals are delivered synchronously
      \begin{itemize}
          \item Example: SIGFPE - floating point errors
          \item Other arithmetic errors
          \item Call comes \textit{right after} you do the incorrect thing
      \end{itemize}
  \end{itemize}
\end{itemize}

\section*{Threading in different languages}
Happy examples: \url{https://gist.github.com/dulimarta/a7e25cb3984e5d2b8e16}
\subsection*{phtr.c (Happy)}
\begin{lstlisting}[language=c]
  #include <pthread.h>
  #include <stdio.h>

  void* happy (void* arg) {
      int* N = (int *)arg;
      for (int k = 0; k < *N; k++)
          printf ("Happy %d\n", k);
      // void* works like java Object. It can be cast to anything
      return (void *) 0xF00D;
  }

  int main () {
      pthread_t happy_id;
      pthread_attr_t happy_attr;

      pthread_attr_init (&happy_attr);
      int num;
      num = 6;
      // Immediately running starting from
      // happy function
      pthread_create (&happy_id, &happy_attr, happy, &num);

      for (int k = 0; k < 20; k++)
          printf ("Before %d\n", k);

      int rc;
      pthread_join (happy_id, (void **) &rc);
      printf ("Thread status is %x\n", rc);

      printf ("End of program\n");

      return 0;
  }
\end{lstlisting}
\subsection*{Happy.cpp}
\begin{lstlisting}[language=c++]
  #include <iostream>
  // Threads are native in C++11, but still need this include
  #include <thread>
  #include <future>

  void happy(int N) {
      for (int k = 0; k < N; k++)
          std::cout << "HAPPY " << k << std::endl;
  }

  int main() {
      // std::async creates a new thread
      // argument to async can be any type, unlike Java or C (void*)
      std::future<void>  happy_end ( std::async(happy, val));

      /* The "grumpy" for-loop in the main thread will run concurrently with
         the "happy" thread */
      for (int k = 0; k < 20; k++) {
          std::cout << "grumpy " << k << std::endl;
      }

      // wait for the "happy" thread to finish
      // .get() will also return a value if thread has a non-void type
      happy_end.get();
      return 0;
  }
\end{lstlisting}

% Wednesday 2016-02-03
\section*{Process/Thread Synchronization}
\begin{itemize}
  \item Consumer must stop if the buffer is empty
    \begin{lstlisting}[language=c]
      /* consumer */
      while (true) {
      	while (counter == 0)
        	/* do nothing */
      	c_item = buff[out];
      	out++;
      	out %= BUFF_SIZE;
      	counter--;
      }
    \end{lstlisting}
  \item Producer may overload the buffer,
  so must stop before the buffer size is full
    \begin{lstlisting}[language=c]
      /* Producer */
      while (true) {
        p_item = produce_item()
        while (counter == BUFF_SIZE) {
        	/* do nothing */
        }
        buff[in] = p_item;
        in %= BUFF_SIZE;
        counter++;
      }
    \end{lstlisting}
\end{itemize}

\subsection*{Interruptible Points $\rightarrow$ Race condition}
Consider that “counter++” and “counter--” may both require 3 assembly instructions each
The update to the shared variable cannot be guaranteed.
\begin{itemize}
  \item Mutual exclusion:
  \begin{itemize}
    \item Only one thread in critical section at a time
  \end{itemize}
  \item Progress
  \begin{itemize}
    \item If a process is waiting, that process can only receive permission to
    enter the \textbf{critical section} by process CS or in exit section
  \end{itemize}
  \item Bounded waiting
  \begin{itemize}
    \item Whenever a process is waiting, at some point
    the process can continue.
    \item Show that no process will dominate – “fair to all processes”
  \end{itemize}
\end{itemize}
\subsection*{Proving Peterson's solution}
\begin{itemize}
  \item How do we prove Mutual Exclusion?
  \begin{enumerate}
    \item turn $\neq j$ OR \textit{flag[i]} is false $\Rightarrow$ turn $i$
    \item \textit{flag[i]} is false OR turn $\neq i \Rightarrow$ turn $j$
    \item CONTRADICTION: There must only be one turn
  \end{enumerate}
  \item How do we prove Progress?
  \begin{itemize}
    \item Flag is set false definitively in exit section
  \end{itemize}
  \item How do we prove Bounded Waiting?
  \begin{itemize}
    \item Turn shows that \textit{no one} progress will dominate
  \end{itemize}
\end{itemize}
\subsection*{Hardware Solutions}
\begin{itemize}
  \item TestAndSet: What is the old value in that variable? And updating the
  variable
  \item CmpAndSwap: Similar to TSL, but update the lock only if condition is
  met
  \item Disable interrupt \textbf{THIS IS NOT A GOOD SOLUTION}
  \item Multiple CPUs: Only disable it for a given CPU but not for the others
  \textbf{NOT A GREAT SOLUTION EITHER}
\end{itemize}
\end{document}
